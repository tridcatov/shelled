\documentclass[14pt]{beamer}
\usepackage[T2A]{fontenc}
\usepackage[utf8]{inputenc}
\usepackage[english,russian]{babel}
\usepackage{amssymb,amsfonts,amsmath,mathtext}
\usepackage{cite,enumerate,float,indentfirst}

\usetheme{Warsaw}
\usecolortheme{whale}

\title{\textbf{Безоблачный умный дом} Cloudless smart home}
\author{Платформа для разработки автоматизированных 
  систем управления устройствами IoT
}
\date{2018}

\begin{document}

\maketitle

\begin{frame}
  \frametitle{Проблемы и вызовы}
  \begin{enumerate}
    \item Управляющее ПО в облаке
    \item Отсутствие полного контроля за устройствами в доме (машине)
    \item Сложность настройки и конфигурации, отстутствие автоматизации
  \end{enumerate}
\end{frame}

\begin{frame}
  \frametitle{Решение}
  \begin{enumerate}
    \item Self Organized System (Plug'n'Play)
    \item Cloudless --- полная независимость от удаленного ПО (boxed solution)
    \item Простое управление и конфигурация через UI
  \end{enumerate}
\end{frame}

\begin{frame}
  \frametitle{Сценарий демонстрации}
  \begin{enumerate}
    \item Автоматическая регистрация устройств в системе
    \item Управление конфигурациями через UI
    \item Измерение температуры в помещении и озвучивание через
      исполняющее устройство (динамик)
    \item Удаленное включение устройства по событиям от сенсоров
  \end{enumerate}
\end{frame}

\begin{frame}
  \frametitle{Отличительные особенности}
  \begin{enumerate}
    \item Упрощение управления и конфигурации устройствами умного
      дома (user friendly)
    \item Удешевление решения за счет использования дешевых
      доступных компонент (commodity hardware)
    \item Просто добавление новых типов устройств, в том числе 
      3rd party
  \end{enumerate}
\end{frame}

\begin{frame}
  \frametitle{Конкурирующие решения}
  \begin{enumerate}
    \item https://www.devicehive.com/
    \item http://creatordev.io/iot-framework
    \item https://www.mainflux.com/
    \item https://thinger.io/  
  \end{enumerate}
\end{frame}

\begin{frame}
  \frametitle{Технологии}
  \begin{enumerate}
    \item Поддержка Plug'n'play, автоматизация подключения и
      настройки новых устройств
    \item Объединение устройств независимо от географического
      положения и доступности интернета (p2p mesh routing)
  \end{enumerate}
\end{frame}

\begin{frame}
  \frametitle{Бизнес-модель и монетизация}
  \begin{enumerate}
    \item Продажа b2b лицензий на dev-kit
    \item Продажа готовых решений (разработка, настройка, установка)
    \item Продажа конечных устройств
    \item Поддержка, обслуживание, расширение технологической
      базы за счет новых устройств
  \end{enumerate}
\end{frame}

\begin{frame}
  \frametitle{Текущий прогресс}
  Было:
  \begin{itemize}
    \item Raspberry Pi: контроллер
    \item STM32: плата сенсоров и управляемых устройств
    \item Linux: ноутбук под управлением ОС linux для развертки
      виртуальных устройств
  \end{itemize}

  Стало:
  \begin{itemize}
    \item GUI для настройки взаимодействий
    \item sdk-kit для создания сервисов (железо + виртуальные)
  \end{itemize}
\end{frame}

\begin{frame}
  \frametitle{Команда}
  \begin{itemize}
    \item Алексей Кулешов
    \item Владислав Волков
    \item Иван Зотов
    \item Илья Бережной
    \item Алексей Борисенко
    \item Анатолий Колодяжный
  \end{itemize}
  \vfill
  \centerline{Спасибо!}
\end{frame}

\end{document}
